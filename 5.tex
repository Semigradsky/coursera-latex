\documentclass[a4paper,12pt]{article}

%%% Работа с русским языком
\usepackage{cmap}					% поиск в PDF
\usepackage{mathtext} 				% русские буквы в формулах
\usepackage[T2A]{fontenc}			% кодировка
\usepackage[utf8]{inputenc}			% кодировка исходного текста
\usepackage[english,russian]{babel}	% локализация и переносы

%%% Дополнительная работа с математикой
\usepackage{amsfonts,amssymb,amsthm,mathtools} % AMS
\usepackage{amsmath}
\usepackage{icomma} % "Умная" запятая
\usepackage{graphicx}  % Для вставки рисунков
\graphicspath{{materials/}{}}  % папки с картинками
\setlength\fboxsep{3pt} % Отступ рамки \fbox{} от рисунка
\setlength\fboxrule{1pt} % Толщина линий рамки \fbox{}
\usepackage{wrapfig} % Обтекание рисунков и таблиц текстом

%%% Работа с таблицами
\usepackage{array,tabularx,tabulary,booktabs} % Дополнительная работа с таблицами

\begin{document}

\begin{wrapfigure}{l}{0.2\linewidth}
	\includegraphics[width=\linewidth]{filename}
	%\caption{Картинка с обтеканием}
\end{wrapfigure}Прослушав данный курс, вы узнаете, как оформить ваши идеи в виде красивого, профессионально сверстанного текста и слайдов презентации. Вы также научитесь быстро и легко набирать даже самые сложные математические формулы. Перед вами откроются возможности cоздания как простых таблиц, так и таблиц со сложной структурой. Оформление библиографии и ссылок на источники также перестанет быть для вас проблемой. Вам больше не нужно будет тратить время на создание содержания вашей работы, списка таблиц и иллюстраций - этот курс научит вас делать это с использованием всего лишь одной команды. Вы узнаете об особенностях работы с документами, набранными с использованием русского языка, а также научитесь использовать систему LaTeX для создания красивой векторной графики. На этом курсе вы поймете, что качественно оформить любой документ, будь то статья в журнале, курсовая работа или годовой отчет, - легко.

\begin{wraptable}{r}{0.4\linewidth}
		\begin{tabular}{|c||r|}
			\hline
			Число слушателей & 100500 \\ \hline
			Signature track  & 1005  \\ \hline
		\end{tabular}
		\caption{Course Data}
\end{wraptable}В начале пятой недели курса вам будет предложено задание по темам первых четырех недель. Это задание - отличная практика и закрепление приобретенных в ходе курса навыков работы в LaTeX. Если вы рассчитываете по итогам курса получить сертификат с отличием, приготовьтесь к тому, что это задание нужно будет обязательно выполнить. На выполнение задания отводится неделя.

В отличие от тестов это задание будет проверять не автоматическая система - его будут проверять сами слушатели курса (ничего сложного, не переживайте). Процедура оценивания работ начнется на шестой неделе курса - в течение этой недели вам нужно будет проверить несколько работ ваших однокурсников и оценить их в соответствии с критериями, которые мы вам предложим.


\end{document} % конец документа