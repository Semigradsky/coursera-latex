\documentclass[a4paper,12pt]{article}

\usepackage[T2A]{fontenc}			% кодировка
\usepackage[utf8]{inputenc}			% кодировка исходного текста
\usepackage[english,russian]{babel}	% локализация и переносы
\usepackage{array,tabularx,tabulary,booktabs} % Дополнительная работа с таблицами
%Нашел на просторах, но что то тут не так как в задании
\usepackage{float,tabularx}
    \newcolumntype{Y}{>{\raggedleft\arraybackslash}X}
    \newcolumntype{Z}{>{\raggedright\arraybackslash}X}
    %Можно убрать и в таблице поставить Х Х Х, тогда віравнивание будет в ширину
\begin{document}
\begin{center}
\begin{tabularx}{\textwidth}[left]{Y|X|Z}
 Сертификат об окончании & Сертификат об окончании с отличием & Подтвержденный сертификат  \\ 
\hline
\hline 
В течение курса студентам будут предложены 4 теста и 1 финальное взаимнооцениваемое задание (peer assessment). Каждый тест составляет 25 \% от итоговой оценки, при этом в формулу ее расчета идут только 3 лучшие теста из 4. Таким образом, тесты в сумме составляют 75 \% от итоговой оценки. Взаимнооцениваемое задание составляет 25 \% от итоговой оценки. Для получения сертификата об окончании необходимо набрать не менее 60 \% от максимально возможного балла. 
 & Для получения необходимо набрать не менее 80 \% от максимально возможного балла.
  & Система оценивания та же. Для получения подтвержденного сертификата необходимо набрать не менее 60 \% от максимально возможного балла, для получения подтвержденного сертификата с отличием (Verified Certificate with Distinction) необходимо набрать не менее 80 \% от максимально возможного балла. \\
\end{tabularx}
\end{center}
\end{document} 